\documentclass[../main.tex]{subfiles}
\begin{document}
% \section{Summary of tkz-base}
\section{tkz-base小结}

% \subsection{Utility of \tkzname{tkz-base}}
\subsection{\tkzname{tkz-base}宏包工具}

% First of all, you don't have to deal with \TIKZ\ the size of the bounding box.
% Early versions of \tkzNamePack{tkz-euclide} did not control the size of the
% bounding box, now the size of the bounding box is limited.
首先,无需处理\TIKZ{}包围盒尺寸,早期的\tkzNamePack{tkz-euclide}宏包未对包围
盒进行控制,现在提供了包围盒设置命令。

% However, it is sometimes necessary to control the size of what will be
% displayed.
然而,有时也需要控制显示尺寸。
% To do this, you need to have prepared the bounding box you are going to work
% in, this is the role of \tkzNamePack{tkz-base} and its main macro
% \tkzNameMacro{tkzInit}. It is recommended to leave the graphic unit equal to 1
% cm. For some drawings, it is interesting to fix the extreme values
% (xmin,xmax,ymin and ymax) and to "clip" the definition rectangle in order to
% control the size of the figure as well as possible.
为此,需要设置工作区域包围盒,这由\tkzNamePack{tkz-base}宏包实现,
该宏包提供的主要命令是\tkzNameMacro{tkzInit},并建议使用1cm为绘图单位。
某些情况下,则需要指定画布大小(xmin、xmax、ymin和ymax),
并使用\enquote{裁剪}矩形尽可能控制图形尺寸。

% The two macros in \tkzNamePack{tkz-base} that are useful for \tkzNamePack{tkz-euclide} are:
\tkzNamePack{tkz-euclide}宏包使用的\tkzNamePack{tkz-base}宏包提供的两个命令是:

\begin{itemize}
   \item \tkzcname{tkzInit}
   \item \tkzcname{tkzClip}
\end{itemize}
\vspace{20pt}

% To this, I added macros directly linked to the bounding box. You can now view
% it, backup it, restore it (see the documentation of \tkzNamePack{tkz-base}
% section Bounding Box).
为实现该功能,\tkzname{tkz-base}宏包提供了一个命令用于操作包围盒,
以查看、备份、恢复包围盒(参见\tkzNamePack{tzk-base}宏包的Bounding Box小节)。

% \subsection{\tkzcname{tkzInit} and \tkzcname{tkzShowBB}}
\subsection{\tkzcname{tkzInit}命令和\tkzcname{tkzShowBB}命令}

% The rectangle around the figure shows you the bounding box.
用图形四周的矩形表示包围盒。

\begin{tkzexample}[latex=8cm,small]
\begin{tikzpicture}
  \tkzInit[xmin=-1,xmax=3,ymin=-1, ymax=3]
  \tkzGrid
  \tkzShowBB[red,line width=2pt]
\end{tikzpicture}
\end{tkzexample}

% \subsection{\tkzcname{tkzClip}}
\subsection{\tkzcname{tkzClip}命令}

% The role of this macro is to \enquote{clip} the initial rectangle so that only the paths
% contained in this rectangle are drawn.
通过对初始绘图矩形的\enquote{裁剪},仅显示指定矩形范围的内容。

\begin{tkzexample}[latex=8cm,small]
\begin{tikzpicture}
  \tkzInit[xmax=4, ymax=3]
  \tkzAxeXY
  \tkzGrid
  \tkzClip
  \draw[red] (-1,-1)--(5,2);
\end{tikzpicture}
\end{tkzexample}

% It is possible to add a bit of space
可以通过命令选项在裁剪区域四周添加指定的空间。

\vspace*{-10pt}

\begin{tkzltxexample}[]
  \tkzClip[space=1]
\end{tkzltxexample}

% \subsection{\tkzcname{tkzClip} and the option \tkzname{space}}
\subsubsection{\tkzcname{tkzClip}命令和\tkzname{space}选项示例}

% This option allows you to add some space around the \enquote{clipped} rectangle.
该选项可以裁剪区域四周添加指定的空间。

\begin{tkzexample}[latex=8cm,small]
\begin{tikzpicture}
  \tkzInit[xmax=4, ymax=3]
  \tkzAxeXY
  \tkzGrid
  \tkzClip[space=1]
  \draw[red] (-1,-1)--(5,2);
\end{tikzpicture}
\end{tkzexample}

% The dimensions of the \enquote{clipped} rectangle are \tkzname{xmin-1},
% \tkzname{ymin-1}, \tkzname{xmax+1} and \tkzname{ymax+1}.
使用\tkzname{space}选项后,\enquote{裁剪}矩形区域大小为:
\tkzname{xmin-1}、\tkzname{ymin-1}、\tkzname{xmax+1}和\tkzname{ymax+1}。

\end{document}
\endinput
