% Chinese
\setmainfont{Times New Roman}
\defaultCJKfontfeatures{Scale=0.9}
\setCJKmainfont[ItalicFont={KaiTi}, BoldFont={Microsoft YaHei}]{KaiTi} %衬线字体 缺省中文字体为宋体
\setCJKsansfont{Microsoft YaHei} %serif是有衬线字体sans serif无衬线字体。
\setCJKmonofont{FangSong} %中文等宽字体

% Korea
% \setCJKfamilyfont{korm}{Gowun Batang}
\setCJKfamilyfont{korm}{NanumMyeongjo}
\newcommand\krt{\CJKfamily{korm}\CJKspace\footnotesize} % footnotesize small normalsize
\newcommand\kr{\CJKfamily{korm}\CJKspace\normalsize} % footnotesize small normalsize

%  NOTE:  XeLaTeX 下需要把全体带圈数字都设置成 Default 类
\xeCJKDeclareCharClass{Default}{"24EA, "2460->"2473, "3251->"32BF}
\newfontfamily\EnclosedNumbers{Source Han Serif CN} % NOTE: adobe
\AtBeginUTFCommand[\textcircled]{\begingroup\EnclosedNumbers}
\AtEndUTFCommand[\textcircled]{\endgroup}

%--------------------添加本地系统内的中文字体---------------------
\setCJKfamilyfont{song}{SimSun}           % 宋体
\newcommand{\song}{\CJKfamily{song}}      % 宋体
\setCJKfamilyfont{kaiti}{KaiTi}           % 楷体
\newcommand{\kaiti}{\CJKfamily{kaiti}}    % 楷体
\setCJKfamilyfont{fsong}{FangSong}        % 仿宋
\newcommand{\fsong}{\CJKfamily{fsong}}    % 仿宋
\setCJKfamilyfont{hei}{SimHei}            % 黑体
\newcommand{\hei}{\CJKfamily{hei}}        % 黑体
\setCJKfamilyfont{yh}{Microsoft YaHei}    % 微软雅黑
\newcommand{\yh}{\CJKfamily{yh}}          % 微软雅黑

%------------------------------设置字体大小------------------------
\newcommand{\chuhao}{\fontsize{42pt}{\baselineskip}\selectfont}     %初号
\newcommand{\xiaochuhao}{\fontsize{36pt}{\baselineskip}\selectfont} %小初号
\newcommand{\yihao}{\fontsize{28pt}{\baselineskip}\selectfont}      %一号
\newcommand{\erhao}{\fontsize{21pt}{\baselineskip}\selectfont}      %二号
\newcommand{\xiaoerhao}{\fontsize{18pt}{\baselineskip}\selectfont}  %小二号
\newcommand{\sanhao}{\fontsize{15.75pt}{\baselineskip}\selectfont}  %三号
\newcommand{\sihao}{\fontsize{14pt}{\baselineskip}\selectfont}      %四号
\newcommand{\xiaosihao}{\fontsize{12pt}{\baselineskip}\selectfont}  %小四号
\newcommand{\wuhao}{\fontsize{10.5pt}{\baselineskip}\selectfont}    %五号
\newcommand{\xiaowuhao}{\fontsize{9pt}{\baselineskip}\selectfont}   %小五号
\newcommand{\liuhao}{\fontsize{7.875pt}{\baselineskip}\selectfont}  %六号
\newcommand{\qihao}{\fontsize{5.25pt}{\baselineskip}\selectfont}    %七号

%------------------------------标题名称中文化---------------------
\renewcommand\abstractname{\kaiti 摘\ 要}
\renewcommand\abstracttextfont{\kaiti}
\setlength\absleftindent{0pt}
\setlength\absrightindent{0pt}
\setlength{\abstitleskip}{-1.5em} % 调整摘要与正文之间的距离,默认为上下加runin参数后变为左右距离
\abslabeldelim{:}

\renewcommand\figurename{\hei 图}
